% Please do not change the document class
\documentclass[11pt]{scrartcl}

% Please do not change these packages
\usepackage[hidelinks]{hyperref}
\usepackage[none]{hyphenat}
\usepackage{setspace}
\doublespace

% You may add additional packages here
\usepackage{amsmath}

% Please include a clear, concise, and descriptive title+
\title{What is stopping game developers from broadening access to their games? }

\date{January 6, 2017}

% Please do not change the subtitle
\subtitle{COMP160 - Software Engineering}

% Please put your student number in the author field
\author{1605913}



\begin{document}

\maketitle

\abstract{ Proposal}

This paper will discuss whether the games industry should worry whether their games can be played by people with disabilities and how this decision will impact the industry. It is a big decision whether to make a game not playable by a certain group of people. A company’s decision to do this is based off two questions, will this investment in making my game accessible return with a profit and how will this investment benefit my company in the long run? \cite{bierre2005game}. These decisions ae made mainly by the marketing department of these companies to target certain groups of people which will make them the most money \cite{kalapanidas2009playmancer}. This is a huge loss because in a US census the percentage of the population which were disabled made up almost a quarter of the population\cite{bierre2005game} (Table 2). Although the reason for this lack of accessibility in games is not without reason, a company has three problems pertaining to this subject. Not being able to receive feedback, not being able to determine in-game responses and not being able to provide input using conventional input devices. These all require considerable investment\cite{yuan2011game}.

\section{Introduction}

This paper will discuss "Should the Accessibility of Games Matter to Companies in the Games Industry? ". This will be achieved by looking at the three most important reasons as to why game developers, do not expand their games to be used by people with disabilities such as blindness and deafness. This will be done by looking at the three most important engineering challenges that deter game developers from broadening access to their games for specific disabilities and how these problems could potentially be overcome.

\section{Developing Games for the Visually Impaired}

In the games industry haptics, have been a very useful resource to developers. Using haptics, a game can be given a more realistic and immersive feeling\cite{orozco2012role}, but that is not where the only use for haptics lies. 
Haptics become incredibly useful for allowing people with visual impairments to play games. Haptics allow a person with a visual impairment to feel the game instead of using visual cues to operate through the game. 
An example of this would be blind hero where, instead of pressing coloured buttons on a controller to match coloured buttons on a screen, the use of a glove with pager motors inside it help to produce stimuli for the participants to play the game.\cite{yuan2008blind}
If a developer decided that they wanted to broaden the accessibility of their game to the visually impaired using haptics, they would need to first check whether their game was suitable for it. This provides a large barrier for a game developer to try and overcome. This is because if a game has a lot of objects that the player needs to follow, using vibrations overload the player with haptic signals and this will make it very difficult for them to play\cite{orozco2012role}. 
It is also a hurdle for developers because finding substitutes for visual representations at the same level as something like graphics for the visually impaired poses a large challenge \cite{yuan2009towards}.
Although making games with content that substitutes visuals is considered very hard it is not impossible. There are a few ways to make a game accessible to those with low level vision these include:
A high contrast mode, this turns part of the 3D graphics into high contrasted black and white colours on important items for gamers with low vision.
Backpack mode, where items are accessed through a voice activated menu with hierarchies.
Game objects, all game items have a voice feedback and 3D icon sounds and Graphics \cite{westin2004game}
Those are some of the ways that were used to make a game accessible to gamers with low vision, the only drawback to this project is that the researchers had to make their own game because they could not find a 3D graphics game that had a sound interface for the blind. From Table 1 in \cite{westin2004game} it is clear to see that the users foudn it very easy to use and that the usuablity was well done.


\section{Developing Games for Gamers with Auditory Disabilities}



\section{}



\section{Conclusion}

\bibliographystyle{ieeetran}
\bibliography{References}

\end{document}